Solar and other renewable power sources are becoming an integral part
of the eletrical grid that the United States relies on.
In the Southwest alone, solar and wind power plants can provide over
12\% of the power demanded on certain days.
While solar power produces fewer emissions and has a lower carbon
footprint than burning fossil fuels, it is highly variable due
primarily to clouds blocking the sun.
This variability must be understood and controlled by electric
utilities that are required to maintain a reliable grid.
Forecasting the irradiance reaching the ground, the primary input to a
solar panel, can help utilities understand and control the
variability.
This dissertation will explore techniques to forecast irradiance that
make use of data from a network of sensors deployed throughout Tucson,
AZ.
The design and deployment of inexpensive sensors used in the network
will be described.
We will present a forecasting technique that uses data from the
sensor network and outperforms a reference persistence forecast for
one minute to two hours in the future.
We will analyze the errors of this technique in depth and suggest ways
to interpret these errors.
Then, we will describe a data assimilation technique that combines
estimates of irradiance derived from satellite images with data from
the sensor network to improve the satellite estimates.
These improved satellite estimates form the base of future work that will
explore generating forecasts while constantly assimilating new data.

%%% Local Variables:
%%% mode: latex
%%% TeX-master: "dissertation"
%%% End:
