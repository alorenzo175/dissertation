\chapter{CONCLUSION}
\label{chap:conc}

This dissertation has described the techniques and limitations of
solar irradiance forecasts that are used to produce operational solar
power forecasts for utilities.
The forecasting techniques rely on data from an irradiance sensor
network.
In order to obtain such data, we designed and deployed inexpensive,
remote irradiance sensors throughout Tucson, AZ.
Using data from these sensors, we produced forecasts that improve upon
a reference by reducing RMSE by 20\% for time horizons from one minute
to two hours.
We also carefully analyzed the errors of these forecasts and described
how a smoother forecast may have smaller errors when insufficient care
is taken when analyzing errors.
This error analysis has improved our understanding of how to judge the
quality of a forecast based on commonly used forecast metrics.
For longer forecast horizons from 30 minutes to six hours, irradiance
estimates derived from satellite images are used.
Initial satellite estimates had large errors, but we used data
assimilation and data from the sensor network to cut some errors in
half.
Along the way, we studied various methods to estimate the correlation
between pixels in satellite irradiance estimates, including a novel
method based on the difference in cloudiness between two pixels.

Next steps include incorporating a cloud advection model into the data
assimilation routine to produce forecasts and to continuously
incorporate new data while retaining prior information.
These WRF forecasts used for day-ahead and longer forecasts could
benefit from incorporating the actual cloud field at the model
intialization and from an ensemble of model runs.
Eventually, a full physics large eddy simulation of the atmosphere may
be required to properly model the dynamics of clouds and to produce
good forecasts from five minutes to seven days in the future.



%%% Local Variables:
%%% mode: latex
%%% TeX-master: "dissertation"
%%% End:
