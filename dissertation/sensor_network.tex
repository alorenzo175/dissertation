\chapter{IRRADIANCE MONITORING NETWORK}
\label{chap:sens_net}

This chapter describes the irradiance monitoring network that was
deployed in Tucson, AZ, and is the basis for much of the forecasting
work in this dissertation.
First, we present some background on why we built the network.
Next, we describe the design of the custom sensors that we deployed as
part of the network.
Then, we discuss how rooftop PV systems can be used as proxies for
irradiance sensors.
Finally, we present possible improvements one might consider for
future networks.

\section{Background}

The initial application of an irradiance monitoring network is laid
out in \cite{Lonij2013}.
Using historical data from 80 residential rooftop PV systems, Lonij
\etal produced intra-hour solar power forecasts that showed skill over
persistence forecasts.
The local electric utility (Tucson Electric Power) was interested in
receiving these short-term forecasts in real-time.
To generate real-time forecasts, data from sensors need to be gathered
in real-time, thus we set out building the irradiance monitoring
network that would report values in near real-time.

\section{Design of Custom Sensors}

why chosen?


\subsection{Photodiode Sensor}
\label{sec:photodiode}
A number of photodiodes were studied to determine a suitable,
inexpesive replacement for a pyranometer.
We tested a clear domed photodiode which we sanded to better diffuse
light, a thin sheet of telfon glued on a photodiode to diffuse light,
and a an unmodified, flat photodiode, among others.
We found that the flat photodiode (Osram BPW34) provided a reasonable
signal level and cosine response, shown in \cref{fig:pdshape}.
This photodiode is sufficient to detech changes in the irradiance from
a clear-sky, as a pyranometer would, which is the main way the network
will be used.

\begin{figure}[h]
\includegraphics[width=\textwidth]{figs/pdvspy.pdf}
\caption[Comparison of photodiode and pyranometer signals]{A
  comparison of the signal from a photodiode and a calibrated
  pyranometer. The photodiode does not exhibit a perfect cosine
  response with a wider peak that decays too quickly. However, the
  photodiode performs well for the main purpose of detecting changes
  in irradiance. Note that the noise in the measurement of the
  photodiode is about double the noise in the pyranometer measurement.}
\label{fig:pdshape}
\end{figure}

\subsection{Hardware}
\label{sec:senshardware}
A custom printed circuit board was designed for the components that
store and send sensor data to a central server every minute.
The circuit diagram for this board is shown in \cref{fig:circuit}.
Design files for the circuit board can be found online~\cite{sensorrepo}.

\begin{sidewaysfigure}[p]
\includegraphics[angle=-90,width=\textwidth]{figs/circuit.pdf}
\caption[Custom sensor circuit diagram]{Circuit diagram for the
  custom, remote irradiance sensors. See \cref{sec:senshardware} for a
  descirpiton of the components.}
\label{fig:circuit}
\end{sidewaysfigure}

The custom sensors are developed around the Olimex
iMX223-OLinuXino-MICRO board.
The OLinuXino was chosen because it consumes little power ($< 1$W) and it
runs a full Linux operating system which allows for development in any
language that can be installed on Linux along with the usual suite of
Linux tools (SSH, Bash, logs).
It is also relatively inexpensive to purchase complete boards, and the
plans are open-source if one desires to build the board themselves.

Data is communicated via GSM using a MULTITECH MTSMC-H5-U SocketModem.
This modem accepts a standard SIM card that is registered with a
cellular data provider.
The modem is connected to the OLinuXino via USB.
WvDial and PPPD are used to setup the connection to the modem and
allow internet access.

Power to the system is provided by a 10W solar panel and a 6Ah
lead-acid battery.
A standard solar charge controller is used limit the current from the
panel to the battery.
The nominal 12V from the battery is routed to the circuit board with
the OLinuXino and modem and converted to 5VDC with a circuit based on
the LM2676 step-down regulator.

The sensor is designed to accept input from either a calibrated
pyranometer (Apogee SP-212) or an inexpensive silicon photodiode
(Osram BPW34) as discussed in \cref{sec:photodiode}.
A trans-impedance amplifier (MCP602) with appropriate gain is used to
convert the current from the photodiode into a measurable voltage.

The voltage from the sensor (or sensor + trans-impedance amplifier) is
converted to a digital signal with the MCP3201 12-bit
analog-to-digital converter (ADC).
This digital signal is then read the OLinuXino at regular intervals
from the GPIO pins.
An additional 4 channel 12-bit ADC (MCP3204) is used to convert other
values such as enclosure temperature (measured by an LM61) and battery
voltage to be read on the OLinuXino GPIO pins for monitoring.
A 4.096V voltage reference (MCP1541) is used by both ADCs.

A fully assembled printed circuit board is shown in
\cref{fig:sensor_board}.

\begin{figure}[ht]
\includegraphics[width=\textwidth]{figs/sensor_board.jpg}
\caption[Assembled sensor circuit board]{A fully assembled printed
  circuit board for a custom sensor. The red board is the OLinuXino
  and the green board on the left is the cell modem with a flexible
  antenna. Through hole components were chosen for easy soldering.}
\label{fig:sensor_board}
\end{figure}

The circuit board is housed in a waterproof box with an air snorkel
and cable nipples to maintain water resistance.
A metal extrusion serves as a mast to mount the photodiode or
pyranmeter.
The solar panel and this mast are attached to the top of the
waterproof box which can then be placed outside.
A photo of the interior of the box with the circuit board and lead
acid battery is shown in \cref{fig:sensor_int}.
An entire completed sensor undergoing testing is shown in
\cref{fig:sensor_outside}.

\begin{figure}[ht]
  \includegraphics[width=\textwidth]{figs/sensor_interior.jpg}
\caption[Interior of the sensor enclosure]{A photo of the interior of
  the sensor enclosure. The air vent and waterproof cable nipples are
  visible on the right side of the box. The 6Ah 12V motorcycle battery
  is shown near the bottom.}
\label{fig:sensor_int}
\end{figure}

\begin{figure}[p]
  \includegraphics[width=\textwidth]{figs/sensor_outside.jpg}
\caption[A complete custom sensor]{A photo of a complete custom
  irradiance sensor as it undergoes testing outside. The photodiode
  sensor can be see mounted in the upper right of the image connected
  via a coax cable to the circuitry inside the box.}
\label{fig:sensor_outside}
\end{figure}

\subsection{Software}

reverse SSH

\subsection{Possible Improvements}
A number of improvements can be made to the sensor design presented in
this section.
First, most sensor failures were a result of the enclosure and
mounting choice.
Since the sensors were simply placed on the ground, they could be
knocked over by animals or flooded during the monsoon season.
To mitigate issues like this, we recommend that sensors be mounted on
a stake.
While this would complicate deployment somewhat, it would likely
prevent sensor failure due to orientation issues (where the solar
panel is not in the sun to power the device) and some water damage.

In addition to a new enclosure and mounting desgin, a number of
improvments in electronics have been made since the sensors were first
developed.
With the rise of the Internet of Things, there now exist numerous
low-power computing devices that could replace the OLinuXino MICRO in
our design.
These new low-power devices now often come with an integrated lithium
battery charge controller.
Using lithium batteries instead of lead-acid will enable smaller,
lighter sensors.

Finally, improvments can be made to the connection to a wireless
network.
A number of M2M devices have been released that enable wireless
connectivity in low-power, integrated devices.
For example, MultiTech now manufactures a device that integrates a
processor running Linux with the wireless network hardware.
These devices may also include GPS receivers enabling precise location
of sensor devices and more accurate time keeping.

\section{Rooftop PV Systems as Sensors}

One major challenge with using rooftop PV systems as sensors is that
the data is often difficult to collect.
One solution we employed is to use the built-in capabilities of some
inverters to send data directly via FTP.
With the help of a local PV system installer, Technicians for
Sustainability, we are collecting 5 minute averaged power data from
over 70 systems in the Tucson area in near real-time.
Since many inverters connect to a home owners network, we also
explored using cheap Linux devices (Raspberry Pi) to communicate with
inverters on the network and upload the data to a central server.

The electric utilities also have access to inverter data, although it
may be delayed by days or weeks and aggregated to daily or longer values.
With an increase in the installation of smart inverters, utilities are
increasingly able to access inverter data in real-time.
With an appropriate data transfer system in place, one can acquire
the rooftop data from the utility, generate a forecast, and send the
forecast back to the utility.

Since irradiance is the primary driver of PV output power, power data
from rooftop PV systems can act as a proxy for irradiance.
When analyzing both power and irradiance data, a units conversion is
likely necessary.
In our work, we choose to convert all irradiance proxy data to
clear-sky index
\begin{equation}
\label{eq:clrind}
k_n(t) = \frac{y_n(t)}{y_n^{clr}(t)}
\end{equation}
where $y_n(t)$ is the measured time-series and $y_n^{clr}(t)$ the
expected time-series for sensor $n$ if the sky were clear.
This approach accounts for differences in systems such as orientation,
peak power, and shading.
Furthermore, it detrends the diurnal cycle in the data.
The clear-sky expectation should account for temperature and aerosol
effects on a given sensor in some way to produce an unbiased clear-sky
index.

\section{Network Deployment}
map

%%% Local Variables:
%%% mode: latex
%%% TeX-master: "dissertation"
%%% End:
